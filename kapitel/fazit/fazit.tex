\section{Fazit}

Zusammenfassend ist Resilienz in Edge Cloud Systemen ein wichtiges anzustrebendes Ziel. Das Aufkommen moderner Services und das damit verbundene Aufkommen von Edge Cloud Services sind der Grund hierfür. 
Der Aufbau von Resilienz ist auch politisch gewünscht, was verschiedene Gesetze zeigen. Eine große Herausforderung für die Resilienz in solchen Systemen ist die Lastverteilung, sowohl zwischen den Netzwerkverbindungen, 
als auch den einzelnen Knoten des Netzwerkes. Ansätze von Software Defined Networking sind gut geeignet, um diese Herausforderung zu bewältigen. 
Mithilfe dieser Ansätze können verschiedene Strategien zur Lastverteilung dynamisch implementiert werden. Dabei ist zwingend zu unterscheiden, in welchem Bereich die Lastverteilung eingesetzt werden soll. 
Strategien zur Lastverteilung zwischen den Servern eines Netzwerkes beugen nur bedingt gegen Überlastung einzelner Verbindungen vor. 
Diese beiden Arten von Lastverteilung sind wichtige Komponenten und müssen in einer Strategie zur Lastverteilung berücksichtigt werden. 
Darüber hinaus bilden allerdings auch Koordinationsprobleme eine Herausforderung für Edge Cloud Systeme da. Beispielsweise führt die Redundanz der SDN-Controller zwar wie gewollt zu höherer Resilienz, 
allerdings auch zu Managementaufwand, das im schlechtesten Fall zu Abschlägen in der Leistung führt. Durch moderne Koordinierungsmethoden, 
beispielsweise bestimmten Algortihmen mit geringem Overhead kann der Koordinierungsaufwand weitestgehend reduziert werden. Ein weiteres Problem ist das der Verbindungsabbrüche zwischen SDN-Controllern und Netzwerkgeräten. 
Der Zusammenhalt des SDN-Netzwerkes macht das System dynamisch, allerdings auch anfälliger für Störungen. Die negativen Auswirkungen einer Störung können minimiert werden, 
indem jedes Netzwerkgerät einen notwendigen Grad an Eigenständigkeit erhält. Diese Eigenständigkeit führt zwar nicht zum optimalen Ergebnis, reicht aber als Übergangslösung in einer Störungsphase, 
wobei die Alternative einen teilweisen oder ganzen Ausfall des Netzwerkes bedeutet. Die behandelten Herausforderungen stellen keine abgeschlossene Liste dar. 
Ein wichtiger Teil der Resilienz ist auch die Widerstandsfähigkeit gegen Cyber-Attacken, diese wurde hier nicht behandelt. Stattdessen wurde sich rein auf die nicht-böswilligen Herausforderungen fokussiert. 
Trotzdem bieten Strategien zur Erhöhung der IT-Sicherheit in Edge Cloud Systemen weitere Forschungsschwerpunkte, die in zukünftigen Arbeiten erarbeitet werden können.
