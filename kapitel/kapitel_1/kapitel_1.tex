\newpage
\section{Theoretische Grundlagen von Edge Cloud Systemen} \label{infos}
Siehe auch Wissenschaftliches Arbeiten~\footcite[\vglf][S. 1]{Balzert.2008}. %ohne textcommands
Damit sollten alle wichtigen Informationen abgedeckt sein ;-)~\footcite[\vglf][\pagef 1]{Balzert.2008} %mit textcommands
Hier gibt es noch ein Beispiel für ein direktes Zitat\footcite[][\pagef 1]{Balzert.2008} %mit textcommands




\subsubsection{Anregungen finden}
\begin{itemize}
\item \href{http://www.diplom.de}{www.diplom.de}
\item \href{http://www.hausarbeiten.de}{www.hausarbeiten.de}
\item Datenbanken aus Tools and Methods
\item etc.
\end{itemize}

\newpage
\subsection{Anfertigungsphase}
Die Anmeldung ist mittlerweile jeden Mittwoch möglich. \ac{ARP}: Adress Resolution Protocol
\ac{DDOS}: Distributed Denial of Service
\ac{DORA}: Digital Operations Resilience Act
\ac{IP}: Internet Protocol
\ac{IT}: Informationstechnologie
\ac{SDN} : Software Defined Networking
\ac{TCP}: Transmission Control Protocol
\ac{UDP}: User Datagram Protocol

\begin{figure}[H]
\caption{FOM-Vorgaben zur Thesis im Online-Campus}
\includegraphics[width=0.9\textwidth]{campusDownload}
\\
\cite[Quelle: Vgl.][]{FOM}
\end{figure}

\subsection{Begriffserklärungen}

Hier kommt das Kapitel hin todo


\subsection{Probleme der Resilienz in Edge Cloud Systemen}

Hier kommt das Kapitel hin todo






